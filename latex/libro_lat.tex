\documentclass[12pt,a4paper]{book} 
\usepackage[utf8]{inputenc} %utf8 acepta entregas ',ñ,etc
                            %inputec: tipo de entrada  
\usepackage[spanish]{babel} %estructuras en español
%modo matematio, fondos y simbols
\usepackage{amsmath}
\usepackage{amsfonts}
\usepackage{amssymb}
%graficos
\usepackage{tikz}
\usetikzlibrary {graphs}
\usepackage{graphicx}
\usepackage{xcolor}
\definecolor{azul}{rgb}{0,0,255}
\usepackage{setspace}
\usepackage[left=2.54 cm, right=2.54 cm, top=2.54 cm, bottom=2.54 cm]{geometry}

%ESTILO DE PAGINA
%E para paginas pares
%O para paginas impares
%L para el lado izquierdo
%C para el centro
%R para el lado derecho
\usepackage{fancyhdr}
\pagestyle{fancy}
\fancyhf{}
\fancyhead[LE,RO]{UNSAAC 2020-I}
\fancyhead[RE,LO]{CIRCAE}
\fancyfoot[CE,CO]{\leftmark}
\fancyfoot[LE,RO]{\thepage}
\renewcommand{\headrulewidth}{2pt}
\renewcommand{\footrulewidth}{1.5pt}

\begin{document}

%crear una hoja en blanco sin numero de pagina
\begin{titlepage}
%centramos el contenido haciendo uso de la funcion center
   \begin{center}
   %nombre de la universidad
     \huge{\textsc{Universidad Nacional de San \\[0.2cm] Antonio Abad del Cusco}}\\
     \vspace{1cm}
     
    %incluimos el logo con la funcion figure
      \begin{figure}[h]
      \centering
      \includegraphics[width=5 cm, height=8cm]{UNSAAC.png}
      \end{figure}
      
      \vspace{3mm}
      
     %nombramos la facultad y escuela profesional
     \large{\textsc{Facultad de Ingeniería Electríca, Electrónica, \\[0.5mm]Informática y Mecánica }}   \\
     \vspace{1mm}
     \textsc{\large {Escuela Profesional de Ingeniería Electrónica}}
     \vspace{6mm}

   %incertamos las lineas haciendo uso de la funcion rule
    
    %color de la linea rule
    %\textcolor{azul}{\rule{\linewidth}{0.7 mm}}
    \rule{\linewidth}{0.7 mm}
      \begin{spacing}{1}
        \LARGE{\textsc{compilacion de procesamiento de imagenes}}
      \end{spacing}
    \rule{\linewidth}{0.7 mm}\\
    \vspace{2mm}
     
   %uso del entorno tabular para la creacion de los ntegrnates
     \begin{tabular}{ll}
       \emph{Integrantes:}\\ [0.3cm]
       \hspace{3cm}   \textsc{Cuba Arenaza,} Kevin  \\
       \hspace{3cm}   \textsc{Cusihuallpa Huamanttupa,} Kevin  \\
       \hspace{3cm}   \textsc{Florez Zela,} Ruben Dario  \\
       \hspace{3cm}   \textsc{Garfias Quispe,} Joseph  \\
       \hspace{3cm}   \textsc{Gutierrez Benito,} Roly Sandro  \\
       

     \end{tabular}
     
     \vspace{4mm}
   \begin{center}
    \textsc{Cusco-Perú}\\
    \textsc{2021}
    \end{center}
     
   \end{center}
\end{titlepage}

\section{Procesamiento de Imagen Digital}
\subsection{Definición de Imagen Digital}
La imagen se puede definir como una función bidimencional \textit{f(x,y)}, donde las coordenadas espaciales son representadas por \textit{(x,y)} y \textit{f(x,y)} que manifiestan el dominio de la intensidad o nivel de gris de la imagen en ese punto (x,y), la imgen digital son cantidades discretas y finitas y hacen uso de los bits \textit{(unidad pequeña de informacion) en \textbf{0 y 1}}.\\

Por lo tanto, \textit{f(x,y)} toma los siguientes valores:
\begin{itemize}
     \item x $\in$ \textit{[0, h-1]}, donde h es la altura de la imagen
     \item y $\in$ \textit{[0, w-1]}, donde w es el ancho de la imgen
     \item f(x,y) $\in$ \textit{[0, L-1]}, donde L=256 (para una imagen de 8 bits)
\end{itemize}

Una imagen digital está compuesta de un número finito de elementos, el cual representan un valor y ubicación. A estos elementos se les conoce elementales de la imagen o \textit{pixels}.\\
El procesamiento de imagen digital tiene el objetivo de mejorar la calidad o facilitar la búsqueda de información, haciendo uso de un ordenador digital (computadora) y métodos digitales.\\
Tipos de imagen digital:
   \begin{itemize}
     \item Imagenes por mapas de bits; formados por puntos llamados \textit{pixels}
     \item Imagen vectorial; imagenes constituidos por trazos geometricos definidos (funciones matemáticas), que determinan sus caracteristicas  
   \end{itemize}    

Un modelo ideal usado en el proceso computarizado son: 
\begin{enumerate}
   \item [\fbox{1.}]{\em Proceso de bajo nivel:}\\
       Consiste en la reducción el ruido, mejora del contraste y nitidez de la imagen.\\ 
       
              %falta insertar cuadro
       \begin{center}
        \fbox{imagen}$ \longrightarrow$ \fbox{imagen}
      
       \end{center}
       
   \item [\fbox{2.}]{\em Proceso de bajo nivel:}
   \\ Implica la segmentación donde se describira los objetos para poder reducirlos a un formulario adecuado para el procesamiento informático y reconocimiento de objetos individuales.
   
   % (entradas son generalmente imágenes pero la salida son atributos extraidos de la imagen como bordes, contornos, y la identidad de los objetos individuales)
   
   \item [\fbox{3.}]{\em Proceso de bajo nivel:}\\
  Se da sentido al conjunto de objetos reconocidos, como en análisis de imágenes y realizar las funciones cognitivas normalmente asociado con la visión humana.
\end{enumerate}
\newpage
\subsection{Imagenes Digitales a Color}
La imagen digital a color de definir de igual manera donde representa tres funciones \textbf{Red, Green y Blue (RGB)} (Rojo, Verde, Azul) estos valores se definen \textit{f(x,y)} de igual manera para imagenes digitales en escala de grises.\\
La de notación de estas funciones es: \textit{fR(x,y)}, \textit{fG(x,y)} y \textit{fB(x,y)}, el dominio de la intensidad varia desde \textbf{\textit{0 (negro)}} hasta \textbf{\textit{255 (blanco)}}, la combinación dependera de que se esta formulando para el procesamiento de imagen.
 
   
\vspace{4mm}
\begin{center}
\tikz {
       \begin{scope}[transparency group]
           \begin{scope}[blend mode=screen]
              \fill[red!90!black] ( 90:.6) circle (1);
              \fill[green!80!black] (210:.6) circle (1);
              \fill[blue!90!black] (330:.6) circle (1);
           
           \end{scope}
       \end{scope}
}
       
\end{center}


\end{document}